\chapter{Refresher on Preparatory Math}

See: machine learning seeds; what are they?

\section{Vector Algebra}
We will typically use vectors that belong to the number set of real numbers; also, we will typically work in a generic nth dimension.
Complex numbers are rarely used rarely used, mainly in the field of ML applied to physics.

Scalar multiplication has different properties depending on the sign of the scalar.
If the scalar is positive, it expands the vector.
If the scalar is negative, it flips components of the original vector.
If the scalar is less than one, it shrinks the vector.

Every vector of $R^n$ can be decomposed as a linear combination of standard bases.
Even if it may seem a really theoretical topic, there are some practical applications that make use of this standard bases: one-hot encoding is the picked examples.

The norm of a vector is its length and represents its distance from the origin. It provides an "induced" order, useful to compare vectors wrt the origin or between them.
Give the norm of a vector, it holds that:
\begin{itemize}
  \item $|v| \Leftrightarrow v=0$
  \item $|\lambda v| = |\lambda| |v|$
  \item $|v + w| \leq |v| + |w|$
\end{itemize}

Beware that cosine similarity is not a distance metric, since it doesn't meet the triangle inequality.

The dot product of two vectors is designated by: $<v, w> = v \cdot w = |v| \cdot |w| \cdot \cos \theta$.
It is a function with the following domain and codomain: $R^n \rightarrow R$

The cross product cannot always be defined, i.e. cannot be defined in every numeric set. The direction can be determined with the right-hand rule.
Indeed, this type of product results in another vector and not a scalar like in the dot product.

The affine variety is an algebraic set such that it is centered in $x_0$ and shifted by $\alpha S$ given a vector $S$. Is it defined by the set of affine tranformations:
$A = x_0 + \alpha S$

Dato un vettore $s \in R^n$ si dice varietà affine di $R^n$ passante per $x_0$ e traslato di $\alpha$ ??? -> TODO

\section{Multivariate Analysis}
The one-side directional derifative of a function f wrt the direction $s$ in $x_0$ is the following limit, if it exists: TODO limit

TODO si ossreva che 


If both limits exist, are finite and equals for each version approaching x_0, the derivative can be defined.

The partial derivative is denoted by ??? and can be computed as follow (follows?): 

The i-th partial derifvative of f in x_0 is defined as follows: ???

The gradient of f in x_0 is the vector of partial derifatives evaluated in the given point x_0: "nabla" operator

In this case the function is differentiable in the given point.
f is (generally) differentiable if the gradient of the function exists for each point of the domain (typically $R$).

A regular function is defined a s funciton whose partial derivatives are smooth for each point ???

See: random forest and smooth function (class $C^1$ and $C^2$)

The Hessian matrix is the matrix of double derivatives with respect to all the possible pairs of indexes (all possible combinations).

Given a function that maps vectors into vectors ($R^n \rightarrow R^m$) the Jacobian matrix of f in a point x: ???

Doubling the number of times the derivation is applied, we obtain a squared gradient o hessian of $x(\alpha)$ along (???) a vector s.
This gives us the curvature of the function along a given vector.

Funzione di Rosenbrock: ???
First step calculate the hessian matrix of the function, set the function in the point equals to one.

\subsection{Optimization}
Linear (objective) function in an n-dimensional space, marked as $l: R^n \rightarrow R$, can be expressed as:
$l(x) = a^T x + b$ where $a$ and $b$ are constants.

Quadratic (objective) function $q: R^n \rightarrow R$:
$q(x) = \frac{1}{2}x^t Q x + b^t x + c$

Let Q be a symmetric matrix of order n, b in n-dimensional space and c scalar constant.
The hessian of the above function is: $\nabla q(x) = Qx + b$

Notice how $\frac{1}{2}$ comes in useful to make derivative coefficients cleaner.

\subsection{Data Fitting}
A least squares fitting is often picked to fit a polynomial on a set of points $(x_i, y_i) \space \forall i=1,...,m$.

Given $y$ (vertical) vector of original data (or observations), a (vertical) vector of coefficients $a$ and the Vandarmonde matrix of terms $V$, we can put the error variable $E$ equals to $y-Va$.
The error is hence minimized in ordert to obtain the best fitting; we look for $min_{a \in R^{n+1}} \|E\|^2 = min_{a \in R^{n+1}} E^T E$
That can be further developed in $min_{a \in R^{n+1}} (y-Va)^T (y-Va) = min_{a \in R^{n+1}} y^T y + a^T V^T V a - 2y^TVa$.

\subsection{Taylor Approximation}
Even in taylor series the quadratic form appears in the formula, the last addend $\frac{1}{2} h^T \nabla^2 f(x) \nabla$ has the same form of $x^t Q x$.

\section{Convex Analysis}
Let $\omega \subset R^N$ such be a convex set for each $(x, y) \in \omega$, thus it holds that $\alpha x + ( 1 - \alpha ) y$

Almost the same meaning is associated to a convex function. A function is said to be convex if and only if $f(\alpha x + ( 1 - \alpha ) y) \leq \alpha f(x) + ( 1 - \alpha ) f(y)$

See: epigraph set

Why convexity is so important? Two fundamental properties can be expressed on $C^1$ or $C^2$ convex function defined on a convex set $\Omega \subset R^n$.

$f: \Omega \rightarrow R$ of class $C^1$:
\begin{itemize}
  \item f is convex if and only if $f(z) \geq f(y) + \nabla f(y)^T (z-y)$. The RHS a Taylor approximation truncated at the first order.
  \item TODO
\end{itemize}

$f: \Omega \rightarrow R$ of class $C^2$:
\begin{itemize}
  \item f is convex if and only if $\nabla^2 f(x)$ is semidefinite positive for each x in the domain ($x^T A x \geq 0 \space \forall x \in R^n$). This can be considered an extension (generalization) of the concavity in $R \rightarrow R$ functions
  \item TODO. Beware that it's an implication and not a double implication.
\end{itemize}

TODO: how quadratic funcion is convex if the quadratic matrix is semidefinite positive or just definite positive.

See: pseudoconvex function

Pseudoconvexity is a looser definition wrt to convexity. Beware that pseudoconvex functions are not always also convex.

\subsection{Minimum Points}
Beware of the difference between minimum point and minimum. The classification enumerates (names-lists) weak local minimum, strong local minimum and global minimum.

When minimum exists? A handful of theorems can ensure that a minimum exists, just by looking at the function:
\begin{itemize}
  \item f continous on a set $\omega \subset R^n$ closed and limited admits a minimum. The problem shiftw towards verifying that the domain is limited.
  \item
\end{itemize}

Eg.: a coercive function is not limited but has a global minimum due to the Wierstrass theorem.

Why do we put such emphasis on convex functions?
when it comes to optimization problems, convex functions ensure that a solution does not exist or, if it exists, that it overlaps with a local minimum.
This implies a strong semplification of the optimization problem, since it guarantees that if a minimum is found, the reasearch is ended.

We can recognize when we are in a point of local minimum of a function if:
\begin{itemize}
  \item if $f$ is of class $C^1$: $\nabla f(x^*) = 0$
  \item if $f$ is of class $C^2$: TODO
\end{itemize}

Notice how above, after the "if" just necessary function are enumerated. These conditions are not sufficient.

Sufficient conditions refer to a function $f$ of class $C^2$ in a neighbor of the function:
\begin{itemize}
  \item $\nabla f(x^*) = 0$
  \item $\nabla^2 f(x) > 0$ definite positive (TODO: check)
\end{itemize}

These conditions are looser if the function is convex. A convex function of class $C^1$ is said to have a global minimum point $x^*$ if and only if: $\nabla f(x^*) = 0$.

TODO: quadratic functions. Quadratic functions are even easier to handle, since we know they are convex, and we explicitly have the matrix of the quadratic form.

